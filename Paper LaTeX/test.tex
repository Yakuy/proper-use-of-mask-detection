\documentclass{article}
% using timesnew roman
\usepackage{mathptmx}

\usepackage{geometry}
 \geometry{
 a4paper,
 left=40mm,
 right=30mm,
 top=30mm,
 bottom=30mm,
 }

\usepackage{float}
\usepackage{graphicx}
\usepackage{indentfirst}
\usepackage{hyperref}

\usepackage{tikz}
\usetikzlibrary{shapes.geometric, arrows}

\tikzstyle{startstop} = [rectangle, rounded corners, minimum width=3cm, minimum height=1cm,text centered, draw=black, fill=red!30]
\tikzstyle{io} = [trapezium, trapezium left angle=70, trapezium right angle=110, minimum width=2.5cm, minimum height=1cm, text centered, draw=black, fill=blue!30, text width=2cm]
\tikzstyle{process} = [rectangle, minimum width=3cm, minimum height=1cm, text centered, draw=black, fill=orange!30]
\tikzstyle{decision} = [diamond, minimum width=3cm, minimum height=1cm, text centered, draw=black, fill=green!30]
\tikzstyle{arrow} = [thick,->,>=stealth]
\tikzstyle{double-arrow} = [thick,<->,>=stealth]

\usepackage[backend=biber]{biblatex}
\addbibresource{daftar_pustaka.bib}
\graphicspath{ {./images/} }
\renewcommand*\contentsname{Daftar Isi}
\renewcommand{\figurename}{Figur}
\renewcommand{\tablename}{Tabel}

\begin{document}
  \begin{titlepage}
    \begin{center}
      
      \null
      {
      	\LARGE \bfseries MAKALAH}\\
      [0.5cm]
      {\Large Sistem Pembantu Deteksi Penggunaan Masker yang Baik dan Benar Menggunakan Object Detection Faster R-CNN}\\
          
      \vspace{2cm}

      \begin{figure}[H]
        \centering
        \includegraphics[width=200px]{/logo/Lambang UGM.jpg}
      \end{figure}
          
      \vspace{3cm}
    
      {\Large 
      Disusun oleh Tim \bfseries Yakuy 2} {\Large :\\
      \vspace{0.5cm}
      Ardacandra Subiantoro (18/427572/PA/18532)\\
      Arief Pujo Arianto (18/430253/PA/18766)\\
      Chrystian (18/430257/PA/18770)\\
      }


      \vspace{2cm}

      {\normalsize \bfseries
      PROGRAM STUDI S1 ILMU KOMPUTER\\
      DEPARTEMEN ILMU KOMPUTER DAN ELEKTRONIKA\\
      FAKULTAS MATEMATIKA DAN ILMU PENGETAHUAN ALAM\\
      UNIVERSITAS GADJAH MADA\\
      YOGYAKARTA\\
      \vspace{0.2cm}
      2020
      }
            
    \end{center}
  \end{titlepage}

  \pagenumbering{gobble}

  \newpage
  \pagenumbering{arabic}
  \textbf{Abstrak.} Pandemi COVID-19 membawa banyak dampak negatif bagi Indonesia dan seluruh dunia, sehingga penting untuk mempunyai sarana untuk mendeteksi dengan cepat dan akurat pasien yang terinfeksi COVID-19. Gambar X-Ray dada pasien dapat membantu untuk mendeteksi apakah pasien terinfeksi COVID-19 atau tidak. Kami melatih model CapsNet untuk dapat mengklasifikasikan X-Ray dada yang terinfeksi dengan virus COVID-19. Model dilatih dengan 6310 gambar X-Ray dada yang terbagi menjadi tiga kelas : Normal, Pneumonia, dan Covid-19. Kelebihan CapsNet dibanding model \textit{Convolutional Neural Network} tradisional adalah antara lain : \textit{viewpoint invariance}, parameter lebih sedikit, dan generalisasi baik pada sudut pandang baru. Hasil yang diperoleh dari model ini adalah akurasi validasi 0.9556 dan akurasi tes 0.9429. Harapan kami adalah model dapat digunakan untuk menyediakan opini kedua untuk memverifikasi hasil dari bentuk-bentuk tes deteksi COVID-19 lain, dan menyediakan sarana untuk mendeteksi COVID-19 di tempat-tempat yang kekurangan alat tes.
  \newpage
  \tableofcontents
  \newpage
  \section{Pendahuluan}
  	\subsection{Latar Belakang}
	  	Meskipun berbagai penanganan sudah dilakukan tetapi hingga hari ini jumlah kasus COVID-19 terus meningkat. Jumlah kasus total COVID-19 di seluruh dunia sudah mencapai lebih dari 48 juta kasus, dengan lebih dari 1,2 juta kematian akibat COVID-19. Kasus COVID-19 di Indonesia sudah mencapai lebih dari 400 ribu dengan lebih dari 14 ribu kematian. Dampak negatif dari pandemi COVID-19 ini sangat terasa di Indonesia. Direktur Jenderal Pajak Kementerian Keuangan (Kemenkeu) Suryo Utomo membagi dampak pandemi COVID-19 menjadi tiga garis besar \cite{zuraya}. Dampak pertama adalah membuat konsumsi rumah tangga atau daya beli yang merupakan penopang 60 persen terhadap ekonomi jatuh cukup dalam. Hal ini dibuktikan dengan data dari BPS yang mencatatkan bahwa konsumsi rumah tangga turun dari 5,02 persen pada kuartal I 2019 ke 2,84 persen pada kuartal I tahun ini. Dampak kedua yaitu pandemi menimbulkan adanya ketidakpastian yang berkepanjangan sehingga investasi ikut melemah dan berimplikasi pada terhentinya usaha. Dampak ketiga adalah seluruh dunia mengalami pelemahan ekonomi sehingga menyebabkan harga komoditas turun dan ekspor Indonesia ke beberapa negara juga terhenti.
	  	
	  	\par Dengan melemahnya ekonomi di Indonesia ini, banyak perusahaan juga sudah mulai untuk membuka kembali bisnis mereka seperti semula. Berbagai upaya pengawasan pun dilakukan agar setiap orang dapat beraktivitas seperti semula dengan aman tanpa takut untuk tertular COVID-19. Tetapi meskipun berbagai cara pengawasan sudah dilakukan, masih banyak orang yang tidak mematuhi aturan dan tetap beraktivitas seperti biasa. Perlunya pengawasan yang lebih ketat lagi adalah salah satu cara agar semua orang yang beraktivitas mematuhi aturan yang berlaku untuk mengurangi resiko terrularnya COVID-19.
  	\subsection{Rumusan Masalah}
  	\subsection{Batasan Masalah}
  		Batasan masalah yang akan kami gunakan adalah sebagai berikut :
  		\begin{itemize}
  			\item Dataset yang akan digunakan dibatasi pada Kaggle Face Mask Detection milik Larxel.
  			\item Jenis gambar yang akan diklasifikasi dibatasi pada memakai masker, tidak memakai masker, dan memakai masker yang tidak benar.
  			\item Metode penambangan data yang akan kami gunakan adalah Faster R-CNN.
  		\end{itemize}
  \section{Tujuan dan Manfaat}
  \section{Metode}
  \section{Hasil}
  \section{Kesimpulan}
  \section{Lampiran}
\end{document}